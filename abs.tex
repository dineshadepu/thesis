%============================= abs.tex================================
\begin{Abstract}
  % This work models a framework to model abrasive water jet machining. Numerical
  % modeling of abrasive water jet machining consists of several multi-physical
  % processes.

  % Mesh based methods are not suitable for problems involving large mesh
  % deformation and free surface flows. They need to be equipped with additional
  % methods such as level set model the free surface flows. Further several rigid
  % bodies flowing in side the flow field is also not favorable to model. As the
  % AWJM involves large deformation, body extrusion mesh based methods are not
  % favorite.

  % Meshless methods are favorable in these areas as they can easily handle the
  % free surface flow, and large mesh distortion is not involved. Specifically we
  % have smoothed particle hydrodynamics in meshless techniques, of great
  % advantage. SPH is successful in simulating elastic dynamic, fluid flows,
  % rigid-fluid, and fluid-structure phenomenon, erosion due to the solid particle
  % impact. Though there is no work done in updated Lagrangian SPH formalism.
  % There is no open-source formulation available out there.

  % We chose an updated Lagrangian framework in SPH to model the full framework.
  % All the physical processes and multiphase processess are developed and
  % validated. A variant of weakly compressible SPH, correct transport velocity
  % formulation, is developed to model the fluid and solid dynamics which
  % eliminates tensile instability and adds corrected terms and shown that
  % improved accuracy in modeling fluid flows. Developed solver is modified to
  % handle the collision among the elastic solids using a contact force model to
  % eliminate the spurious interaction of close by solids and to incorporate
  % friction between the colliding solids. The contact force is inspired from DEM
  % model and can handle collision between arbitrarily shaped solids. The
  % developed solver is coupled with the discrete element method to model the
  % two-way rigid fluid coupling. The above scheme ctvf is coupled to model fluid
  % structure interaction problem. Problems with hs tank and plate and dam
  % breaking flow on to elastic obstacle are studied. A framework including the
  % above is developed an elasto-plastic constitutive model is incorporated to
  % handle the elastic plastic behaviour of the ductile target when modeling the
  % impact due to the projectile. Several analytical, experimental results were
  % considered to validate the proposed schemes.
%
%
%
%
%
\end{Abstract}
%=======================================================================
