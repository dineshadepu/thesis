\chapter{Introduction}
\label{chap:SPH}
Several real-life applications involve fluid dynamics, structural dynamics, and
interaction between these two materials. Examples include fluid-structure
interaction, rigid bodies collision inside fluid flow, rigid bodies - colliding,
deforming, and eroding an elastic-plastic target. In addition, several
industrial applications involve modeling of some or all of the above processes.
Few such process are, material erosion with a high-pressure water jet with
abrasives, and is widely found in the industry applied to design machining of
metals~\parencite{llanto_recent_2021} and composites~\parencite{alberdi_composite_2013}.
Such machining is widely known as abrasive water jet machining. This is used for
cutting metallic sheets, glass, and composite materials for automobiles,
housing, and aircraft industries
\parencite{alberdi_composite_2013,aich_abrasive_2014,llanto_recent_2021},
respectively.
\begin{figure}
  \centering
  \includegraphics[width=0.7\textwidth]{images/intro/images/intro_section/erosion_hand_drawn_2}
  \caption{A schematic of erosion of a ductile solid due to several arbitrarily
    shaped solids impact in a fluid flow.}
\label{fig:intro-big-picture}
\end{figure}
\Cref{fig:intro-big-picture} depicts abrasive water jet machining. From
\Cref{fig:intro-big-picture}, we can see the erosion of a ductile solid due to
the impact of rigid bodies in a fluid flow. The physical processes involved in
modeling this are labeled in \cref{fig:intro-big-picture}. From the numbered
labels, one needs to model the following physics to model a problem such as this,
\begin{enumerate}
\item fluid dynamics,
\item elastic-plastic dynamics,
\item fluid-structure interaction (FSI),
\item free rigid body dynamics,
\item rigid-fluid coupling (RFC) and rigid-rigid interaction and,
\item solid particle erosion (SPE).
\end{enumerate}
An analytical study of such complex problems are impossible as it is highly
nonlinear and involves several different multi-physical processes. Therefore, we
choose numerical modeling due to its flexibility in handling the nonlinearity of
complexity. Mesh-based techniques such as finite element methods (FEM) and
finite volume methods (FVM) are flexible in modeling physics and have been in
the field for the past few decades. However, these mesh-based techniques are not
flexible while modeling many physical processes together. This is due to mesh
distortion and free surface handling. Meshless techniques can naturally handle
free surfaces and large deformation problems. Meshless schemes can be easily
parallelized. Therefore, in the current work, we chose particle-based numerical
methods to model the physical processes involved in handling a problems like
abrasive water jet machining. We use a coupled SPH and DEM formulations to
model the physical processes. We develop a unified framework that can solve
problems involving all such processes using a single framework that runs in
serial and parallel. Test cases with exact analytical solutions and experimental
cases are used to validate the proposed formulations.


% \textbf{Write about SPH and why we choose SPH, about software, updated
%   Lagrangian open source, DEM, runs on all architectures}.


\section{State of the Art and Motivation}
In the next section, we look at the state of art of modeling these physical
processes with SPH and DEM. We review the works in mesh-based techniques and
their drawbacks. We discuss modeling fluid-structure interaction, rigid-fluid
coupling with free surface flows. The advantages of the meshless technique are
reviewed, and we motivate to develop a new solver which can be adapted to the
current formulation.


\subsection{Fluid and Elastic Dynamics in SPH}
The smoothed particle hydrodynamics (SPH) method has been widely applied since
it was originally proposed to simulate hydrodynamic problems in astrophysics
independently by \textcite{lucy77}, and \textcite{monaghan-gingold-stars-mnras-77}. It
has been extensively applied to simulate problems involving
fluids~\parencite{dalrymple2001sph,shao2003incompressible} for both
compressible~\parencite{monaghan-review:2005}, incompressible fluid
flows~\parencite{sph:fsf:monaghan-jcp94,sph:psph:cummins-rudman:jcp:1999} as well as
elastic dynamics problems~\parencite{randles-1996,gray-ed-2001}, fluid-structure
interaction \parencite{khayyer2018enhanced,he2017coupled}, granular physics
\parencite{bui2008lagrangian,bui2021smoothed} among other areas.
\textcite{monaghan2012smoothed} provides a detailed review of SPH and its
applications.

% in addition to a
% variety of other problems~\textcite{rafiee:fsi-2009,khayyer-fsi-2018,
%   sun2021accurate, bui2008lagrangian}.


% The SPH method is a meshless numerical method originally proposed by
% \textcite{gingold1977smoothed} and \textcite{lucy1977numerical} to model
% astrophysics problems. structural dynamics \parencite{randles1996smoothed,dong2016smoothed},

The SPH method is meshless and Lagrangian, and therefore particles move with the
local velocity. This motion can introduce disorder in the particles and thereby
significantly reduce the accuracy of the method. \textcite{acc_stab_xu:jcp:2009}
proposed an approach to shift the particles so as to obtain a uniform
distribution of particles. This significantly improves accuracy and the method
is referred to as the Particle Shifting Technique (PST). Many different kinds of
PST methods are available in the
literature~\parencite{diff_smoothing_sph:lind:jcp:2012,fickian_smoothing_sph:skillen:cmame:2013,huang_kernel_2019,ye2019sph}.
An alternative approach that ensures particle homogenization for incompressible
fluid flow was proposed as the Transport Velocity Formulation
(TVF)~\parencite{Adami2013}. The method introduced an additional stress term to
account for the motion introduced by the particle shifting. The TVF produces
very accurate results but only works for internal flows. \textcite{zhang_hu_adams17}
proposed the Generalized Transport Velocity Formulation (GTVF) thereby allowing
the TVF to be used for free-surface problems as well as elastic dynamics
problems. This allows for a unified treatment of both fluids and solids.
Similarly, \textcite{oger_ale_sph_2016} introduce ideas from a consistent arbitrary
Lagrangian Eulerian formulation for improving the accuracy of SPH. They employ a
Riemann-based formulation to solve fluid mechanics problems and introduce
particle shifting to obtain highly accurate simulations for internal and
free-surface problems. The PST has also been employed in the
context of the $\delta$-SPH schemes\parencite{sun_consistent_2019}.


The Entropically Damped Artificial Compressibility SPH scheme
(EDAC-SPH)~\parencite{edac-sph:cf:2019} introduces an evolution equation for the
pressure and significantly reduces the noise in the pressure since it features
a pressure diffusion term. The approach has a thermodynamic
justification~\parencite{Clausen2013} and produces very accurate
results.


Recently, \textcite{antuono2021delta} carefully combine the ALE-SPH method of
\textcite{oger_ale_sph_2016} and the consistent $\delta$-SPH formulation of
\textcite{sun_consistent_2019} to improve the accuracy of the $\delta$-SPH
method. They show the importance of the additional terms to the accuracy.


Modeling of elastic solids in SPH was first proposed by
\textcite{libersky1991smooth}, where the authors studied the high-speed impact
problems. Despite SPH being applied to elastic solid modeling, it suffers from
tensile instability~\parencite{swegle1995smoothed}. The tensile instability problem
has given rise to the total Lagrangian SPH \parencite{belytschko2000unified,
  bonet2002alternative, vignjevic2006sph}, where the derivatives of particle
properties are computed in the reference configuration using the deformation
gradient. The updated Lagrangian approach is used by \textcite{gray2001sph} and
\textcite{monaghan2000sph} where an artificial stress term is introduced to control
instabilities. Many other variants of the updated Lagrangian SPH have been
proposed. Godunov SPH \parencite{sugiura2017extension} utilizes a Riemann solver to
reduce the usage of artificial viscosity and a new equation of state is
formulated to solve the tensile instability problem. \textcite{dyka1995approach} use
two sets of particles, where on one set of particles the stress is computed, and
the other set of particles are used for the evolution of other properties,
through which the tensile instability has been overcome.
\textcite{zhang2017generalized} extended the transport velocity formulation (TVF) of
\textcite{adami2013transport} to structural dynamics problems and free-surface fluid
mechanics problems. Here the particles are moved with a transport velocity
rather than the momentum velocity to ensure a more homogeneous particle
distribution. This approach also solves the tensile instability problem.

With the notable exception of the GTVF scheme~\parencite{zhang_hu_adams17}, most
other applications of the PST have been in the context of fluid mechanics. The
GTVF method provides a unified approach to solve both weakly-compressible fluids
as well as solids. However, the method suffers from a few issues. In order to
work for free-surface problems the method relies on using a different background
pressure for each particle and introduces a few numerical corrections to work
around issues. For example, the smoothing length of the homogenization force is
different from that used by the other equations and this parameter is somewhat
ad-hoc. For solid mechanics problems the method uses the transport velocity of
the particle rather than the true velocity in order to compute the strain and
rotation tensor. In addition there are some terms in the governing equations
that are ignored which play a major role. We also note that the method is not
robust to a change in the particle homogenization force.

Having discussed state of the art in modeling of fluid and solid dynamics in
SPH, we now discuss the works in the literature on modeling of the collision
between the elastic solids using the SPH method and its drawbacks.


\subsection{Collision SPH}
The dynamics of projectiles can be modeled by assuming the bodies to be
rigid, as done in the discrete element method (DEM) \parencite{zhan2021surface}.
However, this does not consider any elastic or elastic-plastic deformations of
these colliding bodies. In these cases either finite element methods (FEM)
\parencite{rodrigues2019elastic} or meshless methods such as material point
method~\parencite{sulsky1994particle} or smoothed particle hydrodynamics (SPH)
\parencite{gingold1977smoothed} can be used. In this context, meshless methods, such
as the SPH method, are advantageous when modeling large deformation of solids as
they do not suffer from issues such as mesh entanglement.

SPH has been successful in modeling the collision between the elastic as well as
elastic-plastic solids \parencite{gray2001sph, cleary2010elastoplastic}.
\textcite{yan2021simulation} introduced the interfacial SPH scheme, where the SPH
forces are computed only within each body but the interaction between two bodies
is computed using a repulsive force inspired from ABAQUS.
\textcite{vyas2021collisional} has modeled the interaction between a rigid body with
an elastic solid, using a penalty-based contact force model.
\textcite{mohseni2021particle} use a contact model in order to simulate the
collision between non-smooth rigid bodies with elasto-plastic targets. In both
\parencite{vyas2021collisional} and \parencite{mohseni2021particle} the friction between
the rigid body and target is considered.


Though SPH~\parencite{gray2001sph} has been successful in modeling the collision
between the elastic solids, it does not consider friction between the colliding
solids. Another problem is that the model generates spurious forces on bodies
which are moving close to each other (within the influence radius of the SPH
particles) but not actually interacting. \textcite{yan2021simulation} introduced the
interfacial SPH scheme, which eliminates the spurious interactions but it cannot
handle friction between the solids. The interaction force does not consider the
shape of the solids in contact. \textcite{mohseni2021particle} models the
interaction between a rigid body and a ductile solid using a contact force
model, where the bodies are divided as being primary and a secondary. In
\parencite{mohseni2021particle}, the primary body is usually treated as the rigid
body and the body on which the erosion is simulated is treated as secondary.
\textcite{vyas2021collisional} also consider the collision between a rigid and
elastic body and also makes a clear distinction between primary and secondary
bodies. It is not clear what would happen if both bodies were elastic or if
there is no clear way to distinguish between a primary and secondary body. The
methods proposed by \textcite{vyas2021collisional,mohseni2021particle} are not
applied to model the interaction between elastic solids.

We next look at modeling of fluid-structure interaction.


\subsection{Fluid-Structure Interaction}
The interaction between the ductile target and the incoming jet can be
considered as a fluid-structure interaction phenomenon. Mesh-based schemes such
as finite element method (FEM) \parencite{lozovskiy2015unconditionally} and finite
volume method (FVM) \parencite{jasak2007updated} have been used for the last few
decades in modeling the FSI problems. However, mesh-based methods are not
favorable when dealing with free surface flow problems or problems involving
large deformation of the structure. This is due to explicit free surface
tracking, and mesh distortion \parencite{moresi2003lagrangian} while dealing with
large deformation in solids. Therefore, meshless methods are preferred while
handling FSI problems involving free surfaces, multiphase flows, and large
deformation in solids. The smoothed particle hydrodynamics (SPH) and material
point method (MPM) are more commonly used to model the fluid, while the
solids are modeled with SPH or Reproducing Kernel Particle Methods (RKPM), or
the Discrete Element Method (DEM) \parencite{hu2010material,li2022material}. These
meshless techniques have been coupled for the past two decades to model the
fluid-structure interaction. A few schemes with SPH and MPM are SPH-DEM~
\parencite{wu2016coupled}, SPH-total Lagrangian SPH
(TLSPH)~\parencite{salehizadeh2022coupled}, SPH-reproduction kernel particle method
(RKPM)~\parencite{peng2021coupling}, SPH-Peridynamics~\parencite{sun2020smoothed},
MPM-DEM~\parencite{singer2022partitioned}. For more, see the review by
\parencite{khayyer2022systematic}.

Several different meshless techniques are coupled in order to simulate the FSI
phenomenon. FSI is modeled by several works using SPH, such as, WCSPH-Total
Lagrangian SPH (TLSPH) \parencite{zhan2019stabilized}, WCSPH-Updated Lagrangian SPH
(ULSPH) \parencite{antoci2007numerical}, ISPH-TLSPH\parencite{salehizadeh2022coupled}.
However, no work is reported in an updated Lagrangian framework to model FSI
with a transport velocity formulation with corrections included.


We now look at the works which are used to model interaction between fluid and
projectile, where the projectile is a rigid body with six degrees of freedom.
Further, we discuss works that deal with the interaction among rigid bodies and
consider the shape of the solid during the collision.


\subsection{Rigid-Fluid Coupling and Rigid-Rigid Interaction}
If we consider particles in a jet of fluid, we may assume the the incoming
projectiles as rigid bodies. The influence of rigid body on the incoming jet and
vice versa is modeled as rigid fluid coupling. In rigid fluid coupling problems,
mesh-based schemes \parencite{dettmer_computational_2006} have been used in practice
to model fluid dynamics for several decades. However, these schemes are
unfavorable when dealing with free surface flows and mediums undergoing huge
deformation \parencite{walkley_finite_2005}. Among many meshless techniques, moving
particle semi-implicit (MPS) - discrete element method (DEM)
\parencite{guo2017numerical} and smoothed particle hydrodynamics (SPH)-DEM
\parencite{canelas2016sph} are two coupling strategies used to handle rigid-fluid
coupling problems. The rigid body interaction is modeled with DEM. In addition
to DEM, works where SPH \parencite{amicarelli2015smoothed} is used to model the
rigid-rigid interaction. However, it does not incorporate the friction between
the bodies. MPS and SPH are used to model the fluid phase, and a coupling
strategy is utilized to model the interaction between the rigid body and the
fluid phase.

\textcite{cundall_discrete_1979} proposed DEM to study discrete granular materials.
In DEM, the particle dynamics follow simple Newtonian force laws and interact at
their surfaces through a contact force. A shape, mass, and moment of inertia are
assigned to each particle, and originally the particles are assumed to have a
spherical shape. Variations of the DEM were proposed to handle bodies with
different shapes, such as dilated polyhedral DEM \parencite{liu_new_2020}, Fourier
series-based discrete element method \parencite{lai_fourier_2020}
Gilbert-Johnson-Keerthi (GJK)-DEM \parencite{wachs2012grains3d}, discrete function
representation based DEM \parencite{lu2012critical}, level set DEM method
\parencite{duriez2021precision}.

Interaction between bodies with irregular geometries is modeled by multi-sphere
approach \parencite{kruggel-emden_study_2008}, and surface mesh represented
(SMR)-DEM ~\parencite{zhan2021surface}. However, the multi-sphere approach fails to
handle the contact accurately with bodies involving sharp corners, as the force
law assumes the contact between two spherical particles. SMR-DEM requires
additional information to handle the collision, such as connectivity between the
particles comprising the body in addition to the particle positions. However, we
need additional sets of particles to handle the interaction between the rigid
body and the fluid particles. \textcite{mohseni2021particle} proposed a new contact
force law, which handles the collision between the bodies based on particle
positions alone. Here the amount of overlap between the bodies is computed using
an SPH method. Utilizing \parencite{mohseni2021particle} to model contact between
the bodies allows us to handle the interaction between the fluid particles with
the same set of particles in contrast to SMR-DEM ~\textcite{zhan2021surface}.
However, \parencite{mohseni2021particle} is applied to contacts involving rigid and
a ductile solids. \parencite{mohseni2021particle} is not applied to model the
interaction between inelastic rigid bodies, and in problems involving fluids.

We finally discuss solid particle erosion modeling with mesh-based and meshless
techniques in the next section.


\subsection{Solid Particle Erosion}
In theoretical studies, \textcite{finnie1972some}, \textcite{bitter1963study}
studied the surface erosion of ductile and brittle materials. An analytical
expression for the material removal at different angles of impact is derived in
these studies. Further, an analytical study is carried out by
\textcite{hutchings1977erosion}, who provides various results relating the
crater depth, erosion rate, and shape of the impactor particles. In numerical
modeling, mesh based methods are in use for the past few decades in modeling the
solid particle erosion. \textcite{molinari2002study,takaffoli2009finite} have
studied the single impact problem using FEM. Meshless techniques are been used
in studying the erosion process, this is due to its advantage in handling
problems with large deformation. \textcite{dong2016smoothed} uses smoothed
particle hydrodynamics (SPH) to model the erosion of a ductile target.

FEM~\parencite{takaffoli2009finite} is successful in modeling solid particle
erosion, however, it is inefficient when modeling the erosion due to many
particles. Further, FEM is not suitable for modeling multiphysics problems.
\textcite{dong2016smoothed} uses smoothed particle hydrodynamics (SPH) to model the
erosion of a ductile target. However, \textcite{dong2016smoothed} does not consider
the arbitrary shape of the projectile. Further, collision among the projectiles
while interacting with the target is not modeled in SPH. To the authors
knowledge there is no open-source implementation in the literature which can
model the solid particle erosion in SPH.

Having discussed the numerical techniques in SPH and DEM to model the complex
physical processes. In the next section, we now set our objectives to develop a
unified framework which can eliminate several drawbacks in the present numerical
techniques.

\section{Objectives}
We develop an open-source framework to model the complex physics involved in
many applications using SPH and DEM. The following are the key goals of the work.
\begin{itemize}
\item Develop a unified technique in SPH to solve both fluid and solid dynamics
  problems.
\item Handle the collision between the elastic solids using a penalty-based contact force.
\item Handle the collision among the arbitrarily shaped rigid projectiles in fluid
  flow. Solve the two-way coupling between the fluid and the rigid bodies.
\item Develop a fluid-structure interaction solver.
\item Provide an open-source implementation for solid particle erosion in SPH.
\end{itemize}

To this end we propose a scheme which we called Corrected Transport Velocity
Formulation (CTVF) that is inspired by the various recent developments but is
consistent and which works for both solid mechanics and fluid mechanics
problems. We derive the transport velocity equations afresh and note that there
are some important terms that are ignored in earlier approaches using TVF. These
terms are significant and improve the accuracy of the method. Similar to
\textcite{oger_ale_sph_2016,sun_consistent_2019}, we detect the free surface
particles and compute their normals using a simpler and computationally
efficient approach which does not require the computation of eigenvalues. This
allows the method to work with free-surfaces without the introduction of
numerical parameters or a variable background pressure. We employ the EDAC
formulation and show that there are additional correction terms in the EDAC
scheme that should be introduced to improve the accuracy of the method.
Furthermore, we show how the EDAC scheme can be used in the context of solid
mechanics problems. We make use of the particle velocity rather than the
transport velocity to compute the velocity gradient, strain, and rotation rate
tensors. Our method can be used with any PST and we consider the method of
\textcite{sun_consistent_2019} as well as the iterative PST of
\textcite{huang_kernel_2019}. The method is also robust to the choice of the
smoothing kernel. The resulting method works for both weakly-compressible fluids
as well as solids. The new method may be thought of as an improved extension of
the EDAC-SPH method~\parencite{edac-sph:cf:2019} that can be used for free-surface
problems as well as solid mechanics problems.


In the current work, the collision between elastic solids is modeled using a
penalty-based contact force model. Unlike the approach of
\textcite{yan2021simulation}, the proposed contact force model can handle friction
between the solids as well. The bodies themselves are elastic and this is
simulated using the CTVF SPH method \textcite{adepu2021corrected} presented in the
thesis. The penalty-based force considered here is the one proposed by
\textcite{mohseni2021particle}. In the original model proposed by
\textcite{mohseni2021particle}, the contact force is between a primary body and a
secondary body. In \textcite{mohseni2021particle}, the primary body is usually
treated as the rigid body and the body on which the erosion is simulated is
treated as secondary. It is not clear what would happen if both bodies were
elastic or if there is no clear way to distinguish between a primary and
secondary body. We explore the importance of choosing the primary and secondary
body under collision. We also eliminate the spurious interaction forces between
nearby interacting bodies and show how the contact force can be used for elastic
bodies.

In the current work, we handle FSI problems by the CTVF method, where both
fluids and solid phases are modeled using CTVF alone. We couple CTVF with DEM to
model the rigid fluid coupling problems. The fluid phase is handled using the
CTVF, which provides smooth pressure distribution with EDAC formulation and
homogeneous particle distribution, resulting in accurate fluid modeling. While
DEM is used to handle rigid-rigid interactions and applied to 3D problems, and
it is further modified to handle inelastic collisions by introducing a damping
term in the contact force equation. The interaction between the fluid phase and
rigid bodies is handled using the dummy particle approach \parencite{Adami2012}.

We develop a framework to model the solid particle erosion with CTVF method to
model the elastic-plastic behavior of the ductile target. Where, the plastic
behavior is incorporated using a Johson-Cook constitutive model. The interaction
between the arbitrarily shaped solids and the ductile target are modeled with a
modified contact force formulation. We establish the framework such that it can
handle erosion due to multiple impacts. Here, we allow the multiple particles
to also interact among themselves.


\section{Overview of the Thesis}
The current thesis is organized in the following way. In the next chapter,
\cref{chap:ctvf}, we introduces the basic formalism of SPH and develop the
corrected transport velocity formulation and apply it to model the dynamics of
fluids and elastic structures. The new formulation eliminates several issues SPH
faces. \Cref{chap:csph} improves the collision model by incorporating a contact
force model in SPH while the bodies are colliding. This essentially eliminates
spurious interaction between the bodies and incorporates friction between the
interacting bodies. In \Cref{chap:implementation_detail}, we discuss the
implementation details of contact force algorithms and a sub-stepping
integration strategy in PySPH~\parencite{pysph2020}. \Cref{chap:fsi} discusses the
interaction between the elastic structure and fluids. We use CTVF to model both
both fluid and solid materials and handle the interaction between the fluid and
solid using a dummy particle approach. In \cref{chap:rfc}, we couple CTVF with
DEM to model the rigid fluid coupling phenomenon. The interaction between the
rigid bodies is handled with DEM. \Cref{chap:erosion} models the solid particle
erosion of the ductile body due to an impact of a projectile. The contact
force implemented in \cref{chap:csph} is used to handle the interaction between
the rigid body and the ductile solid. Johnson-Cook constitutive model is
utilized to model the plasticity of the target. We finally summarize our
contributions and conclude with future directions in \cref{chap:conclusions}.


% We show how SPH is used to
% solve the fluid and elastic solid problems in updated Lagrangian formalism.
% Several issues of SPH is discussed, tensile instability, particle
% inhomogenization, through numerical examples.
