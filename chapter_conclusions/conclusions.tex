\chapter{Conclusions and Future Work}
\label{chap:conclusions}

\section{Conclusions}
In this thesis, we have developed a particle-based framework to handle different
physical processes arising in problems with complex physics such as abrasive
water jet machining. We have used SPH to model the fluid and elastic-plastic
dynamics, while the interaction between the solid bodies is modeled with DEM. We
have coupled the developed solver, CTVF, to handle fluid and structural dynamics
in SPH to model the fluid-structure interaction problems. The two-way coupling
between the fluid and rigid bodies is modeled with a coupled SPH-DEM
formulation. Solid particle erosion of a ductile solid due to the impact of
multiple arbitrarily shaped solids is modeled with a coupled SPH-DEM solver.


In the current work, to model the fluid and elastic-plastic dynamics, we have
developed a CTVF scheme. The proposed scheme builds on the original TVF scheme
of \textcite{Adami2013} and is an improvement on the GTVF of
\textcite{zhang_hu_adams17}. For smoother pressure distributions, we adapted
EDAC-SPH~\textcite{edac-sph:cf:2019} to the transport velocity formulation. Two PSTs
are used in the current work for the homogenization of the particles. The developed
scheme is applied to the Taylor-Green vortex, lid-driven cavity, and dam-break
problems to verify the improvement due to the correction terms. An improvement in
the accuracy has been found due to the additional correction terms, with a small
computational cost. In solid mechanics, we have employed momentum velocity to
compute the gradients and used EDAC-SPH to evolve the pressure of the solid
particles. We have applied CTVF to the oscillating plate, colliding elastic
rings, and uniaxial compression problems to test the proposed scheme in
eliminating the tensile instability. It is found that the current scheme is able
to handle higher Poisson ratios and can be easily adapted to different particle
shifting techniques without giving rise to tensile instability.

We have incorporated a linear contact force model in the CTVF solver to handle
the collision between the elastic bodies. The collision SPH formulation has
shown that it can eliminate the spurious interactions that arises on the bodies
when modeling with the CTVF. Friction modeling between the elastic solids has
also been considered with the collision SPH formulation. Two and
three-dimensional simulations, a collision between circular bodies, square
bodies, sliding of a cube on a frictional inclined plane, and rebound of a
spherical body show that the proposed formulation is able to handle the
collision among the elastic bodies with friction. We have discussed the
algorithm of contact force formulation in PySPH. This new algorithm to track
pairwise quantities has shown that it can be successfully adapted to serial and
parallel codes.


We have coupled CTVF with a dummy particle approach by \textcite{Adami2012}, to
model the fluid-structure interaction problems. A sub-stepping integrator is
used to update the states of the fluid and the solid particles. We have
validated the coupled scheme by simulating the deformation of an aluminum plate
over a hydrostatic tank and a dam break hitting an elastic obstacle. Further,
the convergence of the scheme is verified by studying the deformation of an
aluminum plate over a hydrostatic tank problem.

We have modeled the two-way coupling of the rigid body and the fluid flow using
a coupled SPH-DEM approach. We have used the contact force model developed to
handle the collision among the elastic solids. From the simulation of a rigid
sliding cube, the collapse of cylinders, and other problems, we can say that the
developed solver is able to handle collision between arbitrarily shaped solids.
We have executed the rising and falling of rigid bodies in a hydrostatic tank to
validate the coupled part of the SPH-DEM solver. We have seen that the PST
formulation of CTVF provided a homogeneous fluid particle distribution with the
solids.


We have developed an open-source framework that can handle the solids particle
erosion of ductile bodies. We have extended the CTVF solver with a Johnson-Cook
constitutive model to model the plastic behavior of the ductile solid. The
interaction between the impacting particles as well as the projectile-target the
impact is modeled using DEM formulation. The framework has been shown to handle
erosion of the ductile solid due to several single-particle impacts and
multi-particle impacts. A study on performance is executed to determine the
effectiveness of the projectile being represented by boundary particles alone
and with full discretization. We have found that impacting particle represented
with boundary particles alone has twofold performance improvement over
representing the full body. The numerical examples show that the current solver
can show the different regimes of the erosion behavior when a particle impacts,
such as chip separation, lip formation, and gouging of the solid.



% ========================================================
% ========================================================
\section{Future Work}\label{conclusions:future_work}
% ========================================================
% ========================================================
% ========================================================
The proposed method, CTVF, has not been applied to multiphase flow problems. We
believe that the study of correction terms introduced in the current work in
transport velocity-based schemes to model multiphase flows will be of
importance. The developed scheme, CTVF, has applied to model isotropic elastic
dynamics in the current work. It can be extended to handle composite
solids. The current FSI solver can be extended to handle the fluid-structure
interaction of anisotropic structures and can be applied to 3D FSI problems.


In the current work, we assumed a linear spring contact force model while
handling the collision between the solid bodies. A non-linear spring force model
can be studied in future work. For future work, we plan to extend the current
solver being extended to handle rigid fluid coupling in multiphase flows.
Cavitation induced due to the entry of rigid objects can be studied. The
the particle-shifting strategy will be helpful in studying the occurrence of
cavitation due to rigid body entries.


A framework to study the solid particle erosion of ductile solids due to multiple
particle impact is developed in the current work. The framework can be easily
extended to study the SPE of brittle solids. The complete framework can be applied
to large-scale industrial problems, such as abrasive water jet machining.

\todoin{Add practical improved}
