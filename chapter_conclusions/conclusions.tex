\chapter{Conclusions and Future Work}
\label{chap:conclusions}

Abrasive water jet machining is widely seen phenomenon. AWJM involves several
physical processes. Fluid, elastic plastic, rigid bodies collision, interaction
between fluid and elastic solids. Meshbased methods are difficult, faces issues
modeling free surface flows, large deformation problems, and multiphysics or
multiphase problems. We have chosen a particle based methods to study the
physics involved in AWJM. Each subprocess is modelled individually and validated
against analytical and experimental results.

We started with SPH to model the elastic dynamics and fluid flow. After basic
SPH introduction is given, it is applied to classic problems of fluid and
elastic dynamics. It is found that SPH suffers from tensile instability,
particle inhomogeneity, high pressure oscillation problems. A corrected
transport velocity formulation in SPH is developed to solve the issues faced by
the classical SPH. While there exists transport velocity formulation, where the
particles are moved with transport velocity rather than momentum velocity. The
corrected transport velocity formulation considers the additional terms missed
by original formulation, thereafter performing better in terms of accuracy of
the fluid simulations and at elimination of tensile instability. Smoother
pressure distributions is achieved with EDAC formulation.

A contact force model based on DEM is incorporated in the developed updated
Lagrangian SPH model (CTVF) to handle the collision between elastic solids.
Collision among frictional solids is modeled with the developed solver. Spurious
interaction between bodies close by but not touching is eliminated. Collision
between bodies of arbitrary shape were simulated to validate the model. Multiple
bodies collision, sliding and rolling body cases are considered to validate the
model.


The developed solver to model fluid and solid dynamics, CTVF, is extended to
solve fluid structure interaction phenomenon. Both the phases are modeled using
a single unified scheme and by modeling hydrostatic water column on a clamped
beam and dam breaking flow onto an elastic plate validated. This essentially
solved the modeling of the interaction between the high speed jet with the
target.

The influence of the jet on the oncoming projectiles is modeled using the
coupled CTVF-DEM approach. Here, the interaction among the rigid bodies is
handled using the discrete element method.

Erosion.

An opensource formulation.

% ========================================================
% ========================================================
\section{Future work}\label{conclusions:future_work}
% ========================================================
% ========================================================

\subsection{CTVF}
The newly proposed method has not been applied to three dimensional problems
or to fluid structure interaction (FSI) problems. We believe that the method
would be easier to use in the context of FSI since it can handle both fluids
and solids in the same formulation. We propose to investigate these in the
future.


\subsection{CSPH}
A non-linear contact force model can be implemented in the future work. The
current work can be easily extended to the modeling of collision between
elastic and elastic-plastic bodies. Also, the collision between the bodies
undergoing breakage can be easily captured with the current framework.


\subsection{FSI}
For the future work, we plan to extend the current solver being extended to handle
the anisotropic structures and 3D FSI problems.


\subsection{RFC}
For future work, we plan to extend the current solver being extended to handle
rigid fluid coupling in multiphase flows. Cavitation induced due to the entry
of rigid objects can be studied. The particle-shifting strategy will be
helpful in studying the occurring of cavitation due to rigid body entries.
Further, the current work can be applied to study biomedical applications,
such as red blood cells transport in a non-newtonian fluid flow, impact of
arbitrarily shaped rigid objects in fluid flow to note a few.


\subsection{Erosion}
For future work, we plan to extend the current solver being extended to handle
rigid fluid coupling in multiphase flows. Cavitation induced due to the entry
of rigid objects can be studied. The particle-shifting strategy will be
helpful in studying the occurring of cavitation due to rigid body entries.
Further, the current work can be applied to study biomedical applications,
such as red blood cells transport in a non-newtonian fluid flow, impact of
arbitrarily shaped rigid objects in fluid flow to note a few.


\subsection{Water jet machining}
For future work, we plan to extend the current solver being extended to handle
rigid fluid coupling in multiphase flows. Cavitation induced due to the entry
of rigid objects can be studied. The particle-shifting strategy will be
helpful in studying the occurring of cavitation due to rigid body entries.
Further, the current work can be applied to study biomedical applications,
such as red blood cells transport in a non-newtonian fluid flow, impact of
arbitrarily shaped rigid objects in fluid flow to note a few.
