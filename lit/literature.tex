
\chapter{Literature Survey}

The bibliographic entries are to be kept in a file named
\verb|<something>.bib|. In this sample report we call it as
\verb|mylit.bib|. This file must be included without the \verb|.bib|
extension in the main file as: \verb|\bibliography{mylit}|.   Open the
file \verb|mylit.bib| to see the format in which the entries are
written. This is written in the Bib\TeX format. Most of the
bibliographic web pages (Scopus, ISI Web) and software (EndNote, etc)
allow you to export bibliographic entries in the Bib\TeX format.

Citations are referred in the text using \verb|\citet| command which produces citations as though they are part of the text.  In order to say
somebody did this work as a part of a line use:
\verb|\textcite{Batzri1973}| have done extensive work on \ldots.  This will produce
\textcite{Batzri1973} have done extensive work on \ldots. Alternately citations can appear in parenthesis.
The command~\verb|\parencite{Batzri1973}| is used to automatically put the
citations in parenthesis.  As an example consider the extensive work
done in the area of book writing \parencite{Sackmann1995a,Boal2012}.

Conferences \parencite{rich-mart92} or collection of work
\parencite{Sackmann1995a} also have special entries.

It is also possible to cite thesis like this:
\textcite{jariwala00,luding94} or just unpublished work from
\textcite{SunHI03}. Some times there are unclassified bibliographic
entries which can be put under ``misc'' \parencite{Smith99}.



%%% Local Variables: 
%%% mode: latex
%%% TeX-master: "../mainrep"
%%% End: 
